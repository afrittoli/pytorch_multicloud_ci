\RequirePackage{pdfmanagement-testphase}
\DeclareDocumentMetadata{}

\ifdefined\DRAFTMODE
  \documentclass[aspectratio=169,11pt,hyperref={colorlinks=true},draft]{beamer}
\else
  \documentclass[aspectratio=169,11pt,hyperref={colorlinks=true}]{beamer}
\fi
\usepackage[utf8]{inputenc}
\usepackage[T1]{fontenc}
\usepackage{fontspec}
\usepackage[absolute,overlay]{textpos}
\usepackage{listingsutf8}
\usepackage{listings-golang}
\usepackage{tikz}
\usetikzlibrary{external}
\tikzexternalize[prefix=tikz-cache/]
\usepackage{color}
\usepackage{fontawesome5}
% \usepackage{svg}
\usepackage{transparent}


\title{Collaborative Infrastructure at Scale: PyTorch's Multi-Cloud CI Model}
\date[25 Feb 2026]{25 Feb 2026 | \faTwitter ~@blackchip76 | \faGithub ~afrittoli}
\author[Andrea Frittoli]{%
  Andrea Frittoli \\
  Developer Advocate \\
  andrea.frittoli@uk.ibm.com \\
}

\usetheme{af}

% Code style
\setlststyle

\lstdefinelanguage{koyaml}{
  keywords={github, com, afrittoli, examples, ms, go, helloworld},
  sensitive=false,
  comment=[l]{\#},
  morestring=[b]',
  morestring=[b]"
}

% Automatic section frame
% \AtBeginSection{\frame{\sectionpage}}

\begin{document}

\begin{frame}
\titlepage{}
\end{frame}
\note[itemize]{
  \item Welcome everyone
  \item 25 min talk + 5 min Q\&A
  \item Focus on collaborative infrastructure model for PyTorch
}

\begin{speakerframe}[af_wind.jpg]{Andrea Frittoli}%
  {%
  \faGithub ~afrittoli | \faLinkedin ~andreafrittoli | \faTwitter ~@blackchip76
  }%
  {%
  \begin{itemize}
    \item{Open Source Advocate @ IBM}
    \item{Lives in Wales, enjoys the wind}
    \item{Multi-Cloud CI Working Group Lead}
    \item{Member of PyTorch Infra Working Group}
    \item{Tekton, CDEvents maintainer}
  \end{itemize}
  }%
\end{speakerframe}

\begin{lpicrblack}[calum-lewis-vA1L1jRTM70-unsplash.jpg]{%
  Photo by \href{https://unsplash.com/@calumlewis}{\underline{Calum Lewis}}, CC0
  }%
  {%
  \tableofcontents
  }%
  {}
  \frametitle{~~~~~~~~~~~~~~~~~~~~~~~~~~~~~~~~~~~~~~~~~~~~~~~~~~~Contents}
\end{lpicrblack}

% SECTION 1: PYTORCH & THE CHALLENGE
\section[The Infrastructure Challenge]{The Infrastructure Challenge}

\begin{sectionwithpiclargecentral}[denys-nevozhai-Zeu57mprpaI-unsplash.jpg]{Photo by \href{https://unsplash.com/@dnevozhai}{\underline{Denys Nevozhai}}, CC0}{1}
\end{sectionwithpiclargecentral}

\begin{grayframe}
  \frametitle{What is PyTorch?}
  
  \begin{columns}[T]
    \column{0.55\textwidth}
    \textbf{A Leading Open Source ML Framework:}
    \begin{itemize}
      \item Researchers, data scientists, ML engineers
      \item Research and production ML workloads
      \item (Distributed) Training, LLM, RL, Inference
      \item Python API, C++ Core (libtorch)
      \item Eager and Graph Mode (Inductor)
      \item Rapid growth and adoption across industry
    \end{itemize}
    
    \column{0.42\textwidth}
    \hspace{0.3cm}
    % Build Once, Run Anywhere Diagram - Top Right
    \begin{tikzpicture}[scale=0.8, every node/.style={scale=0.8}]
      % User icon at top - Font Awesome icon
      \node[circle, fill=custom2!20, minimum size=1cm] (userbg) at (0,3.8) {};
      \node at (0,3.8) {\includegraphics[width=0.5cm]{img/user-icon.pdf}};
      
      % PyTorch layer
      \node[draw, rectangle, rounded corners, fill=custom1!30, minimum width=3.2cm, minimum height=0.75cm, align=center] (pytorch) at (0,2.2) {\textbf{PyTorch}};
      
      % Arrow from user to pytorch
      \draw[->, line width=1.5pt, custom1] (0,3.3) -- (pytorch.north);
      
      % Platform boxes at bottom - increased spacing
      \node[draw, rectangle, rounded corners, fill=custom3!30, minimum width=1.3cm, minimum height=0.55cm, align=center, font=\scriptsize] (cpu) at (-2.2,0.3) {CPU};
      \node[draw, rectangle, rounded corners, fill=custom3!30, minimum width=1.3cm, minimum height=0.55cm, align=center, font=\scriptsize] (gpu) at (-0.7,0.3) {GPU};
      \node[draw, rectangle, rounded corners, fill=custom3!30, minimum width=1.3cm, minimum height=0.55cm, align=center, font=\scriptsize] (tpu) at (0.7,0.3) {TPU};
      \node[draw, rectangle, rounded corners, fill=custom3!30, minimum width=1.3cm, minimum height=0.55cm, align=center, font=\scriptsize] (others) at (2.2,0.3) {Others};
      
      % Arrows from pytorch to platforms
      \draw[->, line width=1pt, custom2] (pytorch.south) -- (cpu.north);
      \draw[->, line width=1pt, custom2] (pytorch.south) -- (gpu.north);
      \draw[->, line width=1pt, custom2] (pytorch.south) -- (tpu.north);
      \draw[->, line width=1pt, custom2] (pytorch.south) -- (others.north);
    \end{tikzpicture}
  \end{columns}
  
  \vspace{0.05\textheight}
  
  % Timeline - Full Width at Bottom
  \begin{center}
  \begin{tikzpicture}[scale=0.9, every node/.style={scale=0.9}]
    % Timeline line
    \draw[line width=2pt, custom1] (0,0) -- (11,0);
    
    % Timeline points and labels
    % 2016
    \fill[custom1] (0,0) circle (4pt);
    \draw[custom1, line width=1.5pt] (0,0) circle (6pt);
    \node[above, align=center, text width=2cm] at (0,0.3) {\textbf{2016}\\Released by Meta};
    
    % 2018
    \fill[custom2] (2.5,0) circle (4pt);
    \draw[custom2, line width=1.5pt] (2.5,0) circle (6pt);
    \node[above, align=center, text width=2cm] at (2.5,0.3) {\textbf{2018}\\PyTorch 1.0};
    
    % 2022
    \fill[custom1] (5.5,0) circle (4pt);
    \draw[custom1, line width=1.5pt] (5.5,0) circle (6pt);
    \node[above, align=center, text width=2.5cm] at (5.5,0.3) {\textbf{2022}\\Foundation\\governance};
    
    % 2023
    \fill[custom2] (8,0) circle (4pt);
    \draw[custom2, line width=1.5pt] (8,0) circle (6pt);
    \node[above, align=center, text width=2cm] at (8,0.3) {\textbf{2023}\\PyTorch 2.0};
    
    % 2024+
    \fill[custom3] (11,0) circle (4pt);
    \draw[custom3, line width=1.5pt] (11,0) circle (6pt);
    \node[above, align=center, text width=2cm] at (11,0.3) {\textbf{2024+}\\Ecosystem\\Growth};
  \end{tikzpicture}
  \end{center}
\end{grayframe}
\note[itemize]{
  \item Brief intro to PyTorch for those unfamiliar
  \item Emphasize its role in ML ecosystem
}

{
\setbeamercolor*{frametitle}{fg=white}
\settextcolour{white}
\setbeamercolor{background canvas}{bg=myblack}
\setbeamertemplate{background}{
  \begin{tikzpicture}
    \useasboundingbox (0,0) rectangle(\the\paperwidth,\the\paperheight);
    \fill[color=myblack] (0, 0.45\paperheight) rectangle (\the\paperwidth, \the\paperheight);
  \end{tikzpicture}
}
\setwhitelogofootlinewithsource{data from hud.pytorch.org}
\begin{frame}[c]
  \frametitle{CI/CD Infrastructure Today}
  \vspace{-0.15\paperheight}
  \begin{columns}[T]
    \column{0.48\textwidth}
    \begin{itemize}
      \item \~{}1M\$ monthly infrastructure costs
      \item \~{}1M hours of compute per month
      \item \~{}100 distinct PRs/day
    \end{itemize}
    
    \column{0.48\textwidth}
    \textbf{~}
    \begin{itemize}
      \item \~{}1.3k Compute-intensive Workflows/day
      \item Growing year over year
    \end{itemize}
  \end{columns}
  \begin{textblock*}{0.85\paperwidth}(0.075\paperwidth,0.44\paperheight)
    \centering
    \tiny
    \code{pytorch/pytorch} repo only, numbers are estimated
  \end{textblock*}

  \vspace{0.15\paperheight}
  \begin{textblock*}{0.85\paperwidth}(0.075\paperwidth,0.48\paperheight)
    \centering
    \includegraphics[width=0.85\paperwidth,height=0.38\paperheight,keepaspectratio]{img/pytorch_ci_jobs_daily.png}
  \end{textblock*}
  \vspace{0.05\paperheight}
\end{frame}
\setblacklogofootline
}
\note[itemize]{
  \item Discuss about the current scale
  \item Explain numbers and source of data
  \item Discuss growth matrix and fleet expansion
}

\begin{tealframe}
  \frametitle{An Expanding Test Matrix}
  
  \begin{textblock*}{0.44\paperwidth}(0.04\paperwidth,0.25\paperheight)
    \centering
    \includegraphics[width=0.44\paperwidth,height=0.65\paperheight,keepaspectratio]{img/test_matrix_diagram.png}
  \end{textblock*}
  
  \begin{textblock*}{0.44\paperwidth}(0.52\paperwidth,0.35\paperheight)
    \centering
    \includegraphics[width=0.44\paperwidth,keepaspectratio]{img/infra-pie.png}
  \end{textblock*}
\end{tealframe}
\note[itemize]{
  \item Show breadth of vendor ecosystem
  \item Not too much detail
  \item Explain test matrix at high level and where it's growing
}

\begin{grayframe}
  \frametitle{The Challenges}
  \textbf{Scale Challenges:}
  \begin{itemize}
    \item \textbf{Engineering:} Managing a Diverse Fleet, Access to Platforms
    \item \textbf{Financial:} Reconcile Community Wishes with Budget Constraints
  \end{itemize}
  {\tiny~\\}
  \textbf{Security Challenges:}
  \begin{itemize}
    \item Infrastructure from public clouds and private pool
    \item Central Administration
    \item Trusted Builds
  \end{itemize}
  {\tiny~\\}
  \textbf{Experience Challenges:}
  \begin{itemize}
    \item Maintain high-quality end-user experience
    \item Preserve contributor workflow
    \item Provide clear path for vendor engagement
  \end{itemize}
\end{grayframe}
\note[itemize]{
  \item Three interconnected challenges
  \item Scale: both people and money
  \item Security: can't just accept any infrastructure
  \item Experience: can't degrade quality while scaling
  \item Need solution that addresses all three
}

% SECTION 2: PYTORCH FOUNDATION & WORKING GROUPS
\section[PyTorch Foundation]{PyTorch Foundation \& Working Groups}

{
\definecolor{pytorchpurple}{HTML}{812CE5}
\definecolor{pytorchmagenta}{HTML}{B932CC}
\definecolor{pytorchred}{HTML}{E12353}
\setbeamertemplate{background}{%
  \begin{tikzpicture}
    \useasboundingbox (0,0) rectangle(\the\paperwidth,\the\paperheight);
    \shade[left color=pytorchpurple, middle color=pytorchmagenta, right color=pytorchred, shading angle=135]
      (0,0) rectangle (\the\paperwidth, \the\paperheight);
  \end{tikzpicture}%
}
\setwhitelogofootline{}
\begin{frame}[t]
  \begin{textblock*}{\paperwidth}(0cm,0.12\paperheight)%
    \begin{beamercolorbox}[wd=\paperwidth]{section page header}%
      \usebeamerfont{largecentral}%
        \color{white}\centering\insertsection%
    \end{beamercolorbox}%
  \end{textblock*}
\end{frame}
\setblacklogofootline
}

\begin{grayframe}
  \frametitle{The PyTorch Foundation}
  \begin{columns}[T]
    \column{0.38\textwidth}
    \begin{itemize}
      \item Part of the Linux Foundation
      \item Neutral home for PyTorch project
      \item Hosts multiple projects
      \item Thought Leadership (GB)
      \item Technical Leadership (TAC)
      \item Marketing \& Outreach
    \end{itemize}
    
    \column{0.60\textwidth}
    \centering
    \includegraphics[width=\textwidth,keepaspectratio]{img/pytorch-foundation.png}
  \end{columns}
\end{grayframe}
\note[itemize]{
  \item Foundation provides neutral ground
  \item Not controlled by single company
  \item Multiple projects under umbrella
  \item Governance structure enables collaboration
}

\begin{4squaresext}{Working Groups}{%
  \textbf{Multi-Cloud CI}\\
  \vspace{0.1em}
  \footnotesize
  Cloud Agnostic Infrastructure:
  \begin{itemize}
    \item CI/CD jobs and Supporting Infra
    \item Metrics and Monitoring
    \item Fleet Management
  \end{itemize}
  Vendor-managed Runner Pools
  }{%
  \textbf{Accelerator Integration}\\
  \vspace{0.1em}
  \footnotesize
  Software Integration:
  \begin{itemize}
    \item Developer Guide
    \item Framework Improvement
  \end{itemize}
  Reference Test Framework\\
  \vspace{0.3em}
  CI Event Relay Infrastructure
}{%
  \textbf{CI Infra}\\
  \vspace{0.1em}
  \footnotesize
  Responsible for the CI/CD infra:
  \begin{itemize}
    \item Develop and maintain tools
    \item Operate the AWS/GitHub fleet
    \item Execute Releases
  \end{itemize}
  }{%
  \textbf{Security}\\
  \vspace{0.1em}
  \footnotesize
  \begin{itemize}
    \item PyTorch Security Triage
    \item Tools and reports
    \item CI/CD Security
  \end{itemize}
}
\end{4squaresext}
\note[itemize]{
  \item Four working groups
  \item Each with its resposabilities
  \item Discuss overalps and collaborations
}

% SECTION 3: TECHNICAL ARCHITECTURE
\section[Technical Architecture]{Technical Architecture}

\begin{sectionwithpiclargecentral}[nadi-whatisdelirium-fZ8uf_L52wg-unsplash.jpg]{Photo by \href{https://unsplash.com/@whatisdelirium}{\underline{Nadi Whatisdelirium}}, CC0}{1}
\end{sectionwithpiclargecentral}

\begin{grayframe}
  \frametitle{Vendor Contribution Models}
  \textbf{Three Ways to Contribute:}
  \begin{itemize}
    \item \textbf{Compute Resources:} Multiple integration levels
    \item \textbf{Engineering Effort:} Integration, maintenance, support
    \item \textbf{Cloud Credits:} Financial contribution for shared infrastructure
  \end{itemize}
  ~\\
  \textbf{Flexible \& Proportional:}
  \begin{itemize}
    \item Vendors choose contribution model that fits them
    \item Contribution proportional to their investment/stake
  \end{itemize}
\end{grayframe}
\note[itemize]{
  \item Three main contribution types
  \item Compute: not just vendor-managed runners (more detail coming)
  \item Engineering: valuable contribution, integration isn't free
  \item Credits: for vendors who prefer financial contribution
  \item Flexibility helps vendors participate
  \item Proportional model ensures fairness
  \item Will detail compute integration levels in technical section
}

\begin{grayframe}
  \frametitle{Cloud-Agnostic CI Design}
  \textbf{Key Principles:}
  \begin{itemize}
    \item No vendor lock-in
    \item Standardized runner interfaces
    \item Portable CI definitions
    \item Multi-cloud orchestration
  \end{itemize}
\end{grayframe}
\note[itemize]{
  \item Architecture must work across AWS, GCP, Azure, on-prem
  \item GitHub Actions as common interface
  \item Self-hosted runners with standard configuration
  \item Project can move workloads between vendors if needed
}

\begin{grayframe}
  \frametitle{Security \& Isolation}
  \textbf{Security Considerations:}
  \begin{itemize}
    \item Vendor-managed runners in isolated environments
    \item No access to project secrets
    \item Network isolation and egress controls
    \item Audit logging for all runner activity
    \item Regular security reviews
  \end{itemize}
\end{grayframe}
\note[itemize]{
  \item Trust but verify model
  \item Vendors manage infrastructure but can't access sensitive data
  \item Network policies prevent data exfiltration
  \item Comprehensive logging for accountability
  \item Security is non-negotiable
}

\begin{grayframe}
  \frametitle{Monitoring \& Observability}
  \textbf{Visibility Across Vendors:}
  \begin{itemize}
    \item Centralized metrics collection
    \item Performance monitoring per vendor
    \item Cost tracking and attribution
    \item SLA monitoring and alerting
    \item Public dashboards for transparency
  \end{itemize}
\end{grayframe}
\note[itemize]{
  \item Need to see what's happening across all vendors
  \item Track performance: build times, success rates
  \item Understand true costs of each platform
  \item Hold vendors accountable to SLAs
  \item Transparency builds trust with community
}

\begin{grayframe}
  \frametitle{Vendor-Managed Runners}
  \textbf{Operational Model:}
  \begin{itemize}
    \item Vendors provision and maintain runners
    \item Standard configuration templates
    \item Automated scaling based on demand
    \item Health checks and auto-remediation
    \item Vendor-specific optimizations allowed
  \end{itemize}
\end{grayframe}
\note[itemize]{
  \item Vendors know their platforms best
  \item Project provides requirements, vendors implement
  \item Auto-scaling reduces waste
  \item Self-healing reduces maintenance burden
  \item Vendors can optimize for their hardware
}

% SECTION 4: GOVERNANCE MODEL
\section[Governance Model]{Governance Model}

\begin{sectionwithpiclargecentral}[thom-milkovic-FTNGfpYCpGM-unsplash.jpg]{Photo by \href{https://unsplash.com/@thommilkovic}{\underline{Thom Milkovic}}, CC0}{1}
\end{sectionwithpiclargecentral}

\begin{grayframe}
  \frametitle{Governance Principles}
  \textbf{Balancing Act:}
  \begin{itemize}
    \item Project autonomy preserved
    \item Vendor participation encouraged
    \item Community control maintained
    \item Transparent decision-making
    \item Fair representation
  \end{itemize}
\end{grayframe}
\note[itemize]{
  \item Critical balance: accept help without losing control
  \item Vendors are stakeholders, not owners
  \item Community has final say on technical decisions
  \item All decisions documented and public
  \item No single vendor can dominate
}

\begin{grayframe}
  \frametitle{Working Group Structure}
  \textbf{Organization:}
  \begin{itemize}
    \item Regular meetings (bi-weekly/monthly)
    \item Vendor representatives + maintainers
    \item Technical subcommittees
    \item Clear escalation paths
    \item Public meeting notes
  \end{itemize}
\end{grayframe}
\note[itemize]{
  \item Structured but not bureaucratic
  \item Mix of vendor engineers and project maintainers
  \item Subcommittees for deep technical work
  \item Issues can be escalated to steering committee
  \item Transparency through public notes
}

\begin{grayframe}
  \frametitle{Vendor Onboarding}
  \textbf{Standardized Process:}
  \begin{itemize}
    \item Technical requirements review
    \item Security assessment
    \item Pilot phase with limited workloads
    \item Performance validation
    \item Full integration after approval
  \end{itemize}
\end{grayframe}
\note[itemize]{
  \item Not just "plug and play"
  \item Vendors must meet technical standards
  \item Security review is mandatory
  \item Start small, prove reliability
  \item Gradual rollout reduces risk
}

\begin{grayframe}
  \frametitle{Handling Conflicts}
  \textbf{When Interests Diverge:}
  \begin{itemize}
    \item Clear decision-making authority
    \item Project maintainers have final say
    \item Vendor concerns heard but not binding
    \item Documented rationale for decisions
    \item Exit strategy for vendors
  \end{itemize}
\end{grayframe}
\note[itemize]{
  \item Conflicts will happen - plan for them
  \item Maintainers represent project interests
  \item Vendors can voice concerns but can't veto
  \item Transparency in decision rationale
  \item Vendors can leave if unhappy (and vice versa)
}

% SECTION 5: RESULTS & LESSONS
\section[Results \& Next Steps]{Results \& Lessons Learned}

\begin{sectionwithpiclargecentral}[julie_falk_flickr_22258190324_6a583208ae_k.png]{Photo by \href{https://www.flickr.com/photos/piper/}{\underline{Julie Falk}}, CC BY-NC 2.0}{1}
\end{sectionwithpiclargecentral}

\begin{grayframe}
  \frametitle{Platform Coverage Expansion}
  \textbf{Achievements:}
  \begin{itemize}
    \item X\% increase in platform coverage
    \item New accelerator support (specific examples)
    \item Reduced time-to-market for new platforms
    \item Improved test reliability
  \end{itemize}
\end{grayframe}
\note[itemize]{
  \item Quantify the success - use real numbers
  \item Specific examples of new platforms added
  \item Vendors can add support faster than before
  \item Better infrastructure = more reliable tests
}

\begin{grayframe}
  \frametitle{Financial Impact}
  \textbf{Sustainability Achieved:}
  \begin{itemize}
    \item Distributed infrastructure costs
    \item Reduced burden on primary sponsor
    \item Predictable scaling model
    \item Long-term financial sustainability
  \end{itemize}
\end{grayframe}
\note[itemize]{
  \item Model works financially
  \item No single entity bears all costs
  \item Can scale without financial crisis
  \item Sustainable for years to come
}

\begin{grayframe}
  \frametitle{Vendor Onboarding Experience}
  \textbf{Feedback from Vendors:}
  \begin{itemize}
    \item Clear expectations and requirements
    \item Reasonable onboarding timeline
    \item Good technical support from maintainers
    \item Fair governance model
    \item Challenges: [specific examples]
  \end{itemize}
\end{grayframe}
\note[itemize]{
  \item Vendors generally positive
  \item Process is clear and documented
  \item Maintainers are helpful
  \item Governance feels fair
  \item Be honest about challenges faced
}

\begin{grayframe}
  \frametitle{Key Lessons Learned}
  \textbf{What Worked:}
  \begin{itemize}
    \item Clear governance from day one
    \item Security-first approach
    \item Flexible contribution models
    \item Transparent communication
  \end{itemize}
  ~\\
  \textbf{What Was Hard:}
  \begin{itemize}
    \item Balancing vendor needs with project needs
    \item Standardization across diverse platforms
  \end{itemize}
\end{grayframe}
\note[itemize]{
  \item Governance clarity prevents conflicts
  \item Security can't be an afterthought
  \item Flexibility helps vendors participate
  \item Transparency builds trust
  \item Balance is ongoing challenge
  \item Standardization requires compromise
}

\begin{grayframe}
  \frametitle{Applicability to Other LF Projects}
  \textbf{Shared Challenges:}
  \begin{itemize}
    \item Accepting vendor infrastructure
    \item Maintaining project neutrality
    \item Ensuring transparency
    \item Achieving financial sustainability
  \end{itemize}
  ~\\
  \textbf{This Model Can Help:}
  \begin{itemize}
    \item Proven governance framework
    \item Technical architecture patterns
    \item Onboarding playbook
  \end{itemize}
\end{grayframe}
\note[itemize]{
  \item Many LF projects face similar issues
  \item Vendor contributions are common need
  \item Neutrality is critical for LF projects
  \item This model is reusable
  \item Share learnings with community
}

\begin{grayframe}
  \frametitle{Future Directions}
  \textbf{What's Next:}
  \begin{itemize}
    \item Expand to more vendors
    \item Improve automation and tooling
    \item Share model with other projects
    \item Refine governance based on experience
    \item Build community of practice
  \end{itemize}
\end{grayframe}
\note[itemize]{
  \item Model is working, now scale it
  \item More automation reduces overhead
  \item Help other projects adopt this
  \item Governance will evolve
  \item Create community around this approach
}

\begin{grayframe}
  \frametitle{Key Takeaways}
  \begin{itemize}
    \item Infrastructure costs can be distributed sustainably
    \item Vendor contributions work with right governance
    \item Project autonomy and vendor participation can coexist
    \item Security and transparency are non-negotiable
    \item Model is applicable to other LF projects
  \end{itemize}
\end{grayframe}
\note[itemize]{
  \item Summarize main points
  \item Emphasize that this works
  \item Governance is key to success
  \item Security and transparency build trust
  \item Encourage others to try this model
}

% SECTION 6: Q&A
\section[Q\&A]{Questions?}

\begin{sectionwithpiclargecentral}[carl-jorgensen-5nrnxx_tWe8-unsplash.jpg]{Brecon Beacons, Wales, Photo by \href{https://unsplash.com/@scamartist}{\underline{Carl Jorgensen}}, CC0}{1}
\end{sectionwithpiclargecentral}

\begin{blackframe}
  \frametitle{References \& Contact}
  \textbf{Resources:}
  \begin{itemize}
    \item PyTorch Multi-Cloud CI Working Group: [URL]
    \item Documentation: [URL]
    \item GitHub: [URL]
  \end{itemize}
  ~\\
  \textbf{Contact:}
  \begin{itemize}
    \item Andrea Frittoli
    \item andrea.frittoli@uk.ibm.com
    \item \faTwitter ~@blackchip76 | \faGithub ~afrittoli
  \end{itemize}
\end{blackframe}
\note[itemize]{
  \item Provide resources for follow-up
  \item Make yourself available for questions
  \item Encourage people to reach out
}

\end{document}
