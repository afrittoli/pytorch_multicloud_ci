\RequirePackage{pdfmanagement-testphase}
\DeclareDocumentMetadata{}

\documentclass[aspectratio=169,11pt,hyperref={colorlinks=true}]{beamer}
\usepackage[utf8]{inputenc}
\usepackage[T1]{fontenc}
\usepackage{fontspec}
\usepackage[absolute,overlay]{textpos}
\usepackage{listingsutf8}
\usepackage{listings-golang}
\usepackage{tikz}
\usepackage{color}
\usepackage{fontawesome5}
\usepackage{svg}
\usepackage{transparent}


\title{Collaborative Infrastructure at Scale: PyTorch's Multi-Cloud CI Model}
\date[25 Feb 2026]{25 Feb 2026 | \faTwitter ~@blackchip76 | \faGithub ~afrittoli}
\author[Andrea Frittoli]{%
  Andrea Frittoli \\
  Developer Advocate \\
  andrea.frittoli@uk.ibm.com \\
}

\usetheme{af}

% Code style
\setlststyle

\lstdefinelanguage{koyaml}{
  keywords={github, com, afrittoli, examples, ms, go, helloworld},
  sensitive=false,
  comment=[l]{\#},
  morestring=[b]',
  morestring=[b]"
}

% Automatic section frame
% \AtBeginSection{\frame{\sectionpage}}

\begin{document}

% Slide 1: Title
\begin{frame}
\titlepage{}
\end{frame}
\note[itemize]{
  \item Welcome everyone
  \item 25 min talk + 5 min Q\&A
  \item Focus on vendor contribution model for infrastructure
}

% Slide 2: Agenda
\begin{grayframe}
  \frametitle{Agenda}
  \begin{itemize}
    \item The Infrastructure Challenge
    \item PyTorch \& Multi-Cloud CI
    \item Technical Architecture
    \item Governance Model
    \item Results \& Lessons Learned
  \end{itemize}
\end{grayframe}
\note[itemize]{
  \item Brief overview of talk structure
  \item 5 main sections covering problem, solution, implementation, governance, and outcomes
}

% SECTION 1: THE CHALLENGE
\section[The Challenge]{The Infrastructure Challenge}

\begin{sectionwithpiclargecentral}[mike-benna-X-NAMq6uP3Q-unsplash.jpg]{Photo by \href{https://unsplash.com/@mbenna}{\underline{Mike Benna}}, CC0}{1}
\end{sectionwithpiclargecentral}

% Slide 3: Scaling Open Source Infrastructure
\begin{grayframe}
  \frametitle{Scaling Open Source Infrastructure}
  \textbf{The Problem:}
  \begin{itemize}
    \item Open source projects grow in complexity
    \item Infrastructure costs scale exponentially
    \item Traditional centralized funding models break down
    \item Testing requirements multiply with platform diversity
  \end{itemize}
\end{grayframe}
\note[itemize]{
  \item As projects mature, they need more testing
  \item More platforms = more infrastructure
  \item Single sponsor model doesn't scale
  \item Example: Testing on CPU, GPU, TPU, custom accelerators
}

% Slide 4: PyTorch's Unique Challenge
\begin{grayframe}
  \frametitle{PyTorch's Unique Challenge}
  \textbf{A Leading ML Framework:}
  \begin{itemize}
    \item Connects ML practitioners with diverse hardware
    \item CPU vendors: Intel, AMD, ARM
    \item Accelerator vendors: NVIDIA, AMD, Intel, Google, AWS, etc.
    \item Each vendor wants public CI to demonstrate support
    \item Project cannot afford testing every platform
  \end{itemize}
\end{grayframe}
\note[itemize]{
  \item PyTorch is hardware-agnostic by design
  \item Vendors have strong incentive to show their platform works
  \item Public CI = marketing + quality assurance
  \item Cost would be prohibitive for single entity
}

% SECTION 2: PYTORCH & MULTI-CLOUD CI
\section[PyTorch Multi-Cloud CI]{PyTorch \& Multi-Cloud CI}

\begin{sectionwithpiclargecentral}[cat-walking-on-stairs.jpg]{Public domain photo, CC0}{1}
\end{sectionwithpiclargecentral}

% Slide 5: The Multi-Cloud CI Working Group
\begin{grayframe}
  \frametitle{The Multi-Cloud CI Working Group}
  \textbf{A New Approach:}
  \begin{itemize}
    \item Formed to address infrastructure sustainability
    \item Vendor contribution model
    \item Distribute costs while preserving project autonomy
    \item Balance vendor participation with community control
  \end{itemize}
\end{grayframe}
\note[itemize]{
  \item Working group brings together vendors and maintainers
  \item Goal: sustainable infrastructure through shared responsibility
  \item Key principle: project maintains control
  \item Not just about money - also about engineering effort
}

% Slide 6: Contribution Models
\begin{grayframe}
  \frametitle{Vendor Contribution Models}
  \textbf{Three Ways to Contribute:}
  \begin{itemize}
    \item \textbf{Compute Resources:} Vendor-managed runners
    \item \textbf{Engineering Effort:} Integration, maintenance, support
    \item \textbf{Cloud Credits:} Financial contribution for shared infrastructure
  \end{itemize}
  ~\\
  \textbf{Proportional to Investment:} Vendors contribute based on their stake
\end{grayframe}
\note[itemize]{
  \item Flexible model accommodates different vendor capabilities
  \item Some vendors have spare compute, others prefer credits
  \item Engineering effort is valuable - integration isn't free
  \item Proportional model ensures fairness
}

% SECTION 3: TECHNICAL ARCHITECTURE
\section[Technical Architecture]{Technical Architecture}

\begin{sectionwithpiclargecentral}[nadi-whatisdelirium-fZ8uf_L52wg-unsplash.jpg]{Photo by \href{https://unsplash.com/@whatisdelirium}{\underline{Nadi Whatisdelirium}}, CC0}{1}
\end{sectionwithpiclargecentral}

% Slide 7: Cloud-Agnostic CI Design
\begin{grayframe}
  \frametitle{Cloud-Agnostic CI Design}
  \textbf{Key Principles:}
  \begin{itemize}
    \item No vendor lock-in
    \item Standardized runner interfaces
    \item Portable CI definitions
    \item Multi-cloud orchestration
  \end{itemize}
\end{grayframe}
\note[itemize]{
  \item Architecture must work across AWS, GCP, Azure, on-prem
  \item GitHub Actions as common interface
  \item Self-hosted runners with standard configuration
  \item Project can move workloads between vendors if needed
}

% Slide 8: Security Architecture
\begin{grayframe}
  \frametitle{Security \& Isolation}
  \textbf{Security Considerations:}
  \begin{itemize}
    \item Vendor-managed runners in isolated environments
    \item No access to project secrets
    \item Network isolation and egress controls
    \item Audit logging for all runner activity
    \item Regular security reviews
  \end{itemize}
\end{grayframe}
\note[itemize]{
  \item Trust but verify model
  \item Vendors manage infrastructure but can't access sensitive data
  \item Network policies prevent data exfiltration
  \item Comprehensive logging for accountability
  \item Security is non-negotiable
}

% Slide 9: Monitoring & Observability
\begin{grayframe}
  \frametitle{Monitoring \& Observability}
  \textbf{Visibility Across Vendors:}
  \begin{itemize}
    \item Centralized metrics collection
    \item Performance monitoring per vendor
    \item Cost tracking and attribution
    \item SLA monitoring and alerting
    \item Public dashboards for transparency
  \end{itemize}
\end{grayframe}
\note[itemize]{
  \item Need to see what's happening across all vendors
  \item Track performance: build times, success rates
  \item Understand true costs of each platform
  \item Hold vendors accountable to SLAs
  \item Transparency builds trust with community
}

% Slide 10: Runner Management
\begin{grayframe}
  \frametitle{Vendor-Managed Runners}
  \textbf{Operational Model:}
  \begin{itemize}
    \item Vendors provision and maintain runners
    \item Standard configuration templates
    \item Automated scaling based on demand
    \item Health checks and auto-remediation
    \item Vendor-specific optimizations allowed
  \end{itemize}
\end{grayframe}
\note[itemize]{
  \item Vendors know their platforms best
  \item Project provides requirements, vendors implement
  \item Auto-scaling reduces waste
  \item Self-healing reduces maintenance burden
  \item Vendors can optimize for their hardware
}

% SECTION 4: GOVERNANCE MODEL
\section[Governance]{Governance Model}

\begin{sectionwithpiclargecentral}[thom-milkovic-FTNGfpYCpGM-unsplash.jpg]{Photo by \href{https://unsplash.com/@thommilkovic}{\underline{Thom Milkovic}}, CC0}{1}
\end{sectionwithpiclargecentral}

% Slide 11: Governance Principles
\begin{grayframe}
  \frametitle{Governance Principles}
  \textbf{Balancing Act:}
  \begin{itemize}
    \item Project autonomy preserved
    \item Vendor participation encouraged
    \item Community control maintained
    \item Transparent decision-making
    \item Fair representation
  \end{itemize}
\end{grayframe}
\note[itemize]{
  \item Critical balance: accept help without losing control
  \item Vendors are stakeholders, not owners
  \item Community has final say on technical decisions
  \item All decisions documented and public
  \item No single vendor can dominate
}

% Slide 12: Working Group Structure
\begin{grayframe}
  \frametitle{Working Group Structure}
  \textbf{Organization:}
  \begin{itemize}
    \item Regular meetings (bi-weekly/monthly)
    \item Vendor representatives + maintainers
    \item Technical subcommittees
    \item Clear escalation paths
    \item Public meeting notes
  \end{itemize}
\end{grayframe}
\note[itemize]{
  \item Structured but not bureaucratic
  \item Mix of vendor engineers and project maintainers
  \item Subcommittees for deep technical work
  \item Issues can be escalated to steering committee
  \item Transparency through public notes
}

% Slide 13: Vendor Onboarding Process
\begin{grayframe}
  \frametitle{Vendor Onboarding}
  \textbf{Standardized Process:}
  \begin{itemize}
    \item Technical requirements review
    \item Security assessment
    \item Pilot phase with limited workloads
    \item Performance validation
    \item Full integration after approval
  \end{itemize}
\end{grayframe}
\note[itemize]{
  \item Not just "plug and play"
  \item Vendors must meet technical standards
  \item Security review is mandatory
  \item Start small, prove reliability
  \item Gradual rollout reduces risk
}

% Slide 14: Conflict Resolution
\begin{grayframe}
  \frametitle{Handling Conflicts}
  \textbf{When Interests Diverge:}
  \begin{itemize}
    \item Clear decision-making authority
    \item Project maintainers have final say
    \item Vendor concerns heard but not binding
    \item Documented rationale for decisions
    \item Exit strategy for vendors
  \end{itemize}
\end{grayframe}
\note[itemize]{
  \item Conflicts will happen - plan for them
  \item Maintainers represent project interests
  \item Vendors can voice concerns but can't veto
  \item Transparency in decision rationale
  \item Vendors can leave if unhappy (and vice versa)
}

% SECTION 5: RESULTS & LESSONS
\section[Results]{Results \& Lessons Learned}

\begin{sectionwithpiclargecentral}[julie_falk_flickr_22258190324_6a583208ae_k.png]{Photo by \href{https://www.flickr.com/photos/piper/}{\underline{Julie Falk}}, CC BY-NC 2.0}{1}
\end{sectionwithpiclargecentral}

% Slide 15: Platform Coverage Expansion
\begin{grayframe}
  \frametitle{Platform Coverage Expansion}
  \textbf{Achievements:}
  \begin{itemize}
    \item X\% increase in platform coverage
    \item New accelerator support (specific examples)
    \item Reduced time-to-market for new platforms
    \item Improved test reliability
  \end{itemize}
\end{grayframe}
\note[itemize]{
  \item Quantify the success - use real numbers
  \item Specific examples of new platforms added
  \item Vendors can add support faster than before
  \item Better infrastructure = more reliable tests
}

% Slide 16: Cost Savings & Sustainability
\begin{grayframe}
  \frametitle{Financial Impact}
  \textbf{Sustainability Achieved:}
  \begin{itemize}
    \item Distributed infrastructure costs
    \item Reduced burden on primary sponsor
    \item Predictable scaling model
    \item Long-term financial sustainability
  \end{itemize}
\end{grayframe}
\note[itemize]{
  \item Model works financially
  \item No single entity bears all costs
  \item Can scale without financial crisis
  \item Sustainable for years to come
}

% Slide 17: Vendor Experience
\begin{grayframe}
  \frametitle{Vendor Onboarding Experience}
  \textbf{Feedback from Vendors:}
  \begin{itemize}
    \item Clear expectations and requirements
    \item Reasonable onboarding timeline
    \item Good technical support from maintainers
    \item Fair governance model
    \item Challenges: [specific examples]
  \end{itemize}
\end{grayframe}
\note[itemize]{
  \item Vendors generally positive
  \item Process is clear and documented
  \item Maintainers are helpful
  \item Governance feels fair
  \item Be honest about challenges faced
}

% Slide 18: Lessons Learned
\begin{grayframe}
  \frametitle{Key Lessons Learned}
  \textbf{What Worked:}
  \begin{itemize}
    \item Clear governance from day one
    \item Security-first approach
    \item Flexible contribution models
    \item Transparent communication
  \end{itemize}
  ~\\
  \textbf{What Was Hard:}
  \begin{itemize}
    \item Balancing vendor needs with project needs
    \item Standardization across diverse platforms
  \end{itemize}
\end{grayframe}
\note[itemize]{
  \item Governance clarity prevents conflicts
  \item Security can't be an afterthought
  \item Flexibility helps vendors participate
  \item Transparency builds trust
  \item Balance is ongoing challenge
  \item Standardization requires compromise
}

% Slide 19: Applicability to Other Projects
\begin{grayframe}
  \frametitle{Applicability to Other LF Projects}
  \textbf{Shared Challenges:}
  \begin{itemize}
    \item Accepting vendor infrastructure
    \item Maintaining project neutrality
    \item Ensuring transparency
    \item Achieving financial sustainability
  \end{itemize}
  ~\\
  \textbf{This Model Can Help:}
  \begin{itemize}
    \item Proven governance framework
    \item Technical architecture patterns
    \item Onboarding playbook
  \end{itemize}
\end{grayframe}
\note[itemize]{
  \item Many LF projects face similar issues
  \item Vendor contributions are common need
  \item Neutrality is critical for LF projects
  \item This model is reusable
  \item Share learnings with community
}

% Slide 20: Future Directions
\begin{grayframe}
  \frametitle{Future Directions}
  \textbf{What's Next:}
  \begin{itemize}
    \item Expand to more vendors
    \item Improve automation and tooling
    \item Share model with other projects
    \item Refine governance based on experience
    \item Build community of practice
  \end{itemize}
\end{grayframe}
\note[itemize]{
  \item Model is working, now scale it
  \item More automation reduces overhead
  \item Help other projects adopt this
  \item Governance will evolve
  \item Create community around this approach
}

% Slide 21: Key Takeaways
\begin{grayframe}
  \frametitle{Key Takeaways}
  \begin{itemize}
    \item Infrastructure costs can be distributed sustainably
    \item Vendor contributions work with right governance
    \item Project autonomy and vendor participation can coexist
    \item Security and transparency are non-negotiable
    \item Model is applicable to other LF projects
  \end{itemize}
\end{grayframe}
\note[itemize]{
  \item Summarize main points
  \item Emphasize that this works
  \item Governance is key to success
  \item Security and transparency build trust
  \item Encourage others to try this model
}

% SECTION 6: Q&A
\section[Q\&A]{Questions?}

\begin{sectionwithpiclargecentral}[carl-jorgensen-5nrnxx_tWe8-unsplash.jpg]{Brecon Beacons, Wales, Photo by \href{https://unsplash.com/@scamartist}{\underline{Carl Jorgensen}}, CC0}{1}
\end{sectionwithpiclargecentral}

% Slide 22: References & Contact
\begin{blackframe}
  \frametitle{References \& Contact}
  \textbf{Resources:}
  \begin{itemize}
    \item PyTorch Multi-Cloud CI Working Group: [URL]
    \item Documentation: [URL]
    \item GitHub: [URL]
  \end{itemize}
  ~\\
  \textbf{Contact:}
  \begin{itemize}
    \item Andrea Frittoli
    \item andrea.frittoli@uk.ibm.com
    \item \faTwitter ~@blackchip76 | \faGithub ~afrittoli
  \end{itemize}
\end{blackframe}
\note[itemize]{
  \item Provide resources for follow-up
  \item Make yourself available for questions
  \item Encourage people to reach out
}

\end{document}
