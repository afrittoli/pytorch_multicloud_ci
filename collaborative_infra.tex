\RequirePackage{pdfmanagement-testphase}
\DeclareDocumentMetadata{}

\ifdefined\DRAFTMODE
  \documentclass[aspectratio=169,11pt,hyperref={colorlinks=true},draft]{beamer}
\else
  \documentclass[aspectratio=169,11pt,hyperref={colorlinks=true}]{beamer}
\fi
\usepackage[utf8]{inputenc}
\usepackage[T1]{fontenc}
\usepackage{fontspec}
\usepackage[absolute,overlay]{textpos}
\usepackage{listingsutf8}
\usepackage{listings-golang}
\usepackage{tikz}
\usetikzlibrary{external}
\tikzexternalize[prefix=tikz-cache/]
\usepackage{color}
\usepackage{colortbl}
\usepackage{fontawesome6}
% \usepackage{svg}
\usepackage{transparent}


\title{Collaborative Infrastructure at Scale: PyTorch's Multi-Cloud CI Model}
\date[25 Feb 2026]{25 Feb 2026 | \faXTwitter ~@blackchip76 | \faGithub ~afrittoli}
\author[Andrea Frittoli]{%
  Andrea Frittoli \\
  Developer Advocate \\
  andrea.frittoli@uk.ibm.com \\
}

\usetheme{af}

% Code style
\setlststyle

\lstdefinelanguage{koyaml}{
  keywords={github, com, afrittoli, examples, ms, go, helloworld},
  sensitive=false,
  comment=[l]{\#},
  morestring=[b]',
  morestring=[b]"
}

% Automatic section frame
% \AtBeginSection{\frame{\sectionpage}}

\begin{document}

\begin{frame}
\titlepage{}
\end{frame}
\note[itemize]{
  \item Welcome everyone
  \item 25 min talk + 5 min Q\&A
  \item Focus on collaborative infrastructure model for PyTorch
}

\begin{speakerframe}[af_wind.jpg]{Andrea Frittoli}%
  {%
  \faGithub ~afrittoli | \faLinkedin ~andreafrittoli | \faXTwitter ~@blackchip76
  }%
  {%
  \begin{itemize}
    \item{Open Source Advocate @ IBM}
    \item{Lives in Wales, enjoys the wind}
    \item{Multi-Cloud CI Working Group Lead}
    \item{Member of PyTorch Infra Working Group}
    \item{Tekton, CDEvents maintainer}
  \end{itemize}
  }%
\end{speakerframe}

\begin{lpicrblack}[calum-lewis-vA1L1jRTM70-unsplash.jpg]{%
  Photo by \href{https://unsplash.com/@calumlewis}{\underline{Calum Lewis}}, CC0
  }%
  {%
  \tableofcontents
  }%
  {}
  \frametitle{~~~~~~~~~~~~~~~~~~~~~~~~~~~~~~~~~~~~~~~~~~~~~~~~~~~Contents}
\end{lpicrblack}

% SECTION 1: PYTORCH & THE CHALLENGE
\section[The Infrastructure Challenge]{The Infrastructure Challenge}

\begin{sectionwithpiclargecentral}[universal_upscale_0_42586ee4-8ead-4a8f-98bd-fe9c040f6591_0.jpg]{leonardo.ai, CC0}{1}
\end{sectionwithpiclargecentral}

\begin{grayframe}
  \frametitle{What is PyTorch?}
  
  \begin{columns}[T]
    \column{0.55\textwidth}
    \textbf{A Leading Open Source ML Framework:}
    \begin{itemize}
      \item Researchers, data scientists, ML engineers
      \item Research and production ML workloads
      \item (Distributed) Training, LLM, RL, Inference
      \item Python API, C++ Core (libtorch)
      \item Eager and Graph Mode (Inductor)
      \item Rapid growth and adoption across industry
    \end{itemize}
    
    \column{0.42\textwidth}
    \hspace{0.3cm}
    % Build Once, Run Anywhere Diagram - Top Right
    \begin{tikzpicture}[scale=0.8, every node/.style={scale=0.8}]
      % User icon at top - Font Awesome icon
      \node[circle, fill=custom2!20, minimum size=1cm] (userbg) at (0,3.8) {};
      \node at (0,3.8) {\includegraphics[width=0.5cm]{img/user-icon.pdf}};
      
      % PyTorch layer
      \node[draw, rectangle, rounded corners, fill=custom1!30, minimum width=3.2cm, minimum height=0.75cm, align=center] (pytorch) at (0,2.2) {\textbf{PyTorch}};
      
      % Arrow from user to pytorch
      \draw[->, line width=1.5pt, custom1] (0,3.3) -- (pytorch.north);
      
      % Platform boxes at bottom - increased spacing
      \node[draw, rectangle, rounded corners, fill=custom3!30, minimum width=1.3cm, minimum height=0.55cm, align=center, font=\scriptsize] (cpu) at (-2.2,0.3) {CPU};
      \node[draw, rectangle, rounded corners, fill=custom3!30, minimum width=1.3cm, minimum height=0.55cm, align=center, font=\scriptsize] (gpu) at (-0.7,0.3) {GPU};
      \node[draw, rectangle, rounded corners, fill=custom3!30, minimum width=1.3cm, minimum height=0.55cm, align=center, font=\scriptsize] (tpu) at (0.7,0.3) {TPU};
      \node[draw, rectangle, rounded corners, fill=custom3!30, minimum width=1.3cm, minimum height=0.55cm, align=center, font=\scriptsize] (others) at (2.2,0.3) {Others};
      
      % Arrows from pytorch to platforms
      \draw[->, line width=1pt, custom2] (pytorch.south) -- (cpu.north);
      \draw[->, line width=1pt, custom2] (pytorch.south) -- (gpu.north);
      \draw[->, line width=1pt, custom2] (pytorch.south) -- (tpu.north);
      \draw[->, line width=1pt, custom2] (pytorch.south) -- (others.north);
    \end{tikzpicture}
  \end{columns}
  
  \vspace{0.05\textheight}
  
  % Timeline - Full Width at Bottom
  \begin{center}
  \begin{tikzpicture}[scale=0.9, every node/.style={scale=0.9}]
    % Timeline line
    \draw[line width=2pt, custom1] (0,0) -- (11,0);
    
    % Timeline points and labels
    % 2016
    \fill[custom1] (0,0) circle (4pt);
    \draw[custom1, line width=1.5pt] (0,0) circle (6pt);
    \node[above, align=center, text width=2cm] at (0,0.3) {\textbf{2016}\\Released by Meta};
    
    % 2018
    \fill[custom2] (2.5,0) circle (4pt);
    \draw[custom2, line width=1.5pt] (2.5,0) circle (6pt);
    \node[above, align=center, text width=2cm] at (2.5,0.3) {\textbf{2018}\\PyTorch 1.0};
    
    % 2022
    \fill[custom1] (5.5,0) circle (4pt);
    \draw[custom1, line width=1.5pt] (5.5,0) circle (6pt);
    \node[above, align=center, text width=2.5cm] at (5.5,0.3) {\textbf{2022}\\Foundation\\governance};
    
    % 2023
    \fill[custom2] (8,0) circle (4pt);
    \draw[custom2, line width=1.5pt] (8,0) circle (6pt);
    \node[above, align=center, text width=2cm] at (8,0.3) {\textbf{2023}\\PyTorch 2.0};
    
    % 2024+
    \fill[custom3] (11,0) circle (4pt);
    \draw[custom3, line width=1.5pt] (11,0) circle (6pt);
    \node[above, align=center, text width=2cm] at (11,0.3) {\textbf{2024+}\\Ecosystem\\Growth};
  \end{tikzpicture}
  \end{center}
\end{grayframe}
\note[itemize]{
  \item Brief intro to PyTorch for those unfamiliar
  \item Emphasize its role in ML ecosystem
}

{
\setbeamercolor*{frametitle}{fg=white}
\settextcolour{white}
\setbeamercolor{background canvas}{bg=myblack}
\setbeamertemplate{background}{
  \begin{tikzpicture}
    \useasboundingbox (0,0) rectangle(\the\paperwidth,\the\paperheight);
    \fill[color=myblack] (0, 0.45\paperheight) rectangle (\the\paperwidth, \the\paperheight);
  \end{tikzpicture}
}
\setwhitelogofootlinewithsource{data from hud.pytorch.org}
\begin{frame}[c]
  \frametitle{CI/CD Infrastructure Today}
  \vspace{-0.15\paperheight}
  \begin{columns}[T]
    \column{0.48\textwidth}
    \begin{itemize}
      \item \~{}1M\$ monthly infrastructure costs
      \item \~{}1M hours of compute per month
      \item \~{}100 distinct PRs/day
    \end{itemize}
    
    \column{0.48\textwidth}
    \textbf{~}
    \begin{itemize}
      \item \~{}1.3k Compute-intensive Workflows/day
      \item Growing year over year
    \end{itemize}
  \end{columns}
  \begin{textblock*}{0.85\paperwidth}(0.075\paperwidth,0.44\paperheight)
    \centering
    \tiny
    \code{pytorch/pytorch} repo only, numbers are estimated
  \end{textblock*}

  \vspace{0.15\paperheight}
  \begin{textblock*}{0.85\paperwidth}(0.075\paperwidth,0.48\paperheight)
    \centering
    \includegraphics[width=0.85\paperwidth,height=0.38\paperheight,keepaspectratio]{img/pytorch_ci_jobs_daily.png}
  \end{textblock*}
  \vspace{0.05\paperheight}
\end{frame}
\setblacklogofootline
}
\note[itemize]{
  \item Discuss about the current scale
  \item Explain numbers and source of data
  \item Discuss growth matrix and fleet expansion
}

\begin{tealframe}
  \frametitle{An Expanding Test Matrix}
  
  \begin{textblock*}{0.44\paperwidth}(0.04\paperwidth,0.25\paperheight)
    \centering
    \includegraphics[width=0.44\paperwidth,height=0.65\paperheight,keepaspectratio]{img/test_matrix_diagram.png}
  \end{textblock*}
  
  \begin{textblock*}{0.44\paperwidth}(0.52\paperwidth,0.35\paperheight)
    \centering
    \includegraphics[width=0.44\paperwidth,keepaspectratio]{img/infra-pie.png}
  \end{textblock*}
\end{tealframe}
\note[itemize]{
  \item Show breadth of vendor ecosystem
  \item Not too much detail
  \item Explain test matrix at high level and where it's growing
}

\begin{grayframe}
  \frametitle{The Challenges}
  \textbf{Community Challenges:}
  \begin{itemize}
    \item Maintain high-quality end-user experience
    \item Preserve contributor workflow
    \item Provide clear path for vendor engagement
  \end{itemize}
  {\tiny~\\}
  \textbf{Scale Challenges:}
  \begin{itemize}
    \item \textbf{Engineering:} Managing a Diverse Fleet, Access to Platforms
    \item \textbf{Financial:} Reconcile Community Wishes with Budget Constraints
  \end{itemize}
  {\tiny~\\}
  \textbf{Security Challenges:}
  \begin{itemize}
    \item Trust and Consistency in a diverse environment
    \item Central Administration
    \item Trusted Builds
  \end{itemize}
\end{grayframe}
\note[itemize]{
  \item With the expanion of the fleet:
  \item Three interconnected challenges
  \item Scale: both people and money
  \item Security: can't just accept any infrastructure
  \item Community: can't degrade quality while scaling
  \item Explain what I mean by community challenges
  \item Need solution that addresses all three
  \item Consistency may not be the right word, I mean ensuring things are not compromised or altered
}

% SECTION 2: PYTORCH FOUNDATION & WORKING GROUPS
\section[PyTorch Foundation]{PyTorch Foundation \& Working Groups}

{
\definecolor{pytorchpurple}{HTML}{812CE5}
\definecolor{pytorchmagenta}{HTML}{B932CC}
\definecolor{pytorchred}{HTML}{E12353}
\setbeamertemplate{background}{%
  \begin{tikzpicture}
    \useasboundingbox (0,0) rectangle(\the\paperwidth,\the\paperheight);
    \shade[left color=pytorchpurple, middle color=pytorchmagenta, right color=pytorchred, shading angle=135]
      (0,0) rectangle (\the\paperwidth, \the\paperheight);
  \end{tikzpicture}%
}
\setwhitelogofootline{}
\begin{frame}[t]
  \begin{textblock*}{\paperwidth}(0cm,0.12\paperheight)%
    \begin{beamercolorbox}[wd=\paperwidth]{section page header}%
      \usebeamerfont{largecentral}%
        \color{white}\centering\insertsection%
    \end{beamercolorbox}%
  \end{textblock*}
\end{frame}
\setblacklogofootline
}

\begin{grayframe}
  \frametitle{The PyTorch Foundation}
  \begin{columns}[T]
    \column{0.38\textwidth}
    \begin{itemize}
      \item Part of the Linux Foundation
      \item Neutral home for PyTorch project
      \item Hosts multiple projects
      \item Thought Leadership (GB)
      \item Technical Leadership (TAC)
      \item Marketing \& Outreach
    \end{itemize}
    
    \column{0.60\textwidth}
    \centering
    \includegraphics[width=\textwidth,keepaspectratio]{img/pytorch-foundation.png}
  \end{columns}
\end{grayframe}
\note[itemize]{
  \item Foundation provides neutral ground
  \item Not controlled by single company
  \item Multiple projects under umbrella
  \item Governance structure enables collaboration
}

\begin{4squaresext}{Working Groups}{%
  \textbf{Multi-Cloud CI}\\
  \vspace{0.1em}
  \footnotesize
  Cloud Agnostic Infrastructure:
  \begin{itemize}
    \item CI/CD jobs and Supporting Infra
    \item Metrics and Monitoring
    \item Fleet Management
  \end{itemize}
  Vendor-managed Runner Pools
  }{%
  \textbf{Accelerator Integration}\\
  \vspace{0.1em}
  \footnotesize
  Software Integration:
  \begin{itemize}
    \item Developer Guide
    \item Framework Improvement
  \end{itemize}
  Reference Test Framework\\
  \vspace{0.3em}
  CI Event Relay Infrastructure
}{%
  \textbf{CI Infra}\\
  \vspace{0.1em}
  \footnotesize
  Responsible for the CI/CD infra:
  \begin{itemize}
    \item Develop and maintain tools
    \item Operate the AWS/GitHub fleet
    \item Execute Releases
  \end{itemize}
  }{%
  \textbf{Security}\\
  \vspace{0.1em}
  \footnotesize
  \begin{itemize}
    \item PyTorch Security Triage
    \item Tools and reports
    \item CI/CD Security
  \end{itemize}
}
\end{4squaresext}
\note[itemize]{
  \item Four working groups
  \item Each with its resposabilities
  \item Discuss overalps and collaborations
}

% SECTION 3: GOVERNANCE MODEL
\section[Governance Model]{Governance Model}

\begin{sectionwithpiclargecentral}[khampha-phimmachak-IRUcIOMOiEI-unsplash.jpg]{Photo by \href{https://unsplash.com/@khampha}{\underline{Khampha Phimmachak}}, CC0}{1}
\end{sectionwithpiclargecentral}

\begin{stripedframe}{Personas: A Balancing Act}{%
  \textbf{End Users}
  \vspace{0.3em}
  \begin{itemize}
    \item Stable software
    \item Available on my dev platform
    \item Available on my prod platform
    \item Following the industry at speed
  \end{itemize}
}{%
  \textbf{Platform Vendors}
  \vspace{0.3em}
  \begin{itemize}
    \item PyTorch for my platform
    \item Demonstrate platform compatibility
    \item Low bar to contribution of infrastructure
    \item Clear guidance and processes
  \end{itemize}
}{%
  \textbf{Project Maintainers}
  \vspace{0.3em}
  \begin{itemize}
    \item Project Success
    \item Stay relevant for end-users
    \item High quality secure builds
    \item Manageable CI/CD System (scale)
    \item Smooth CI/CD experience for contributors
  \end{itemize}
}{%
  \textbf{PyTorch Foundation}
  \vspace{0.3em}
  \begin{itemize}
    \item Ensure Vendor Neutrality
    \item Vendor participation encouraged
    \item Fair representation
    \item Financial sustainability
  \end{itemize}
}
\end{stripedframe}
\note[itemize]{
  \item Project autonomy preserved
  \item Vendor participation encouraged
  \item Community control maintained
  \item Transparent decision-making
  \item Fair representation
  \item Critical balance: accept help without losing control
  \item Vendors are stakeholders, not owners
  \item Community has final say on technical decisions
  \item All decisions documented and public
  \item No single vendor can dominate
}

\begin{grayframe}
  \frametitle{Integration Models}
  \textbf{Three Ways to Contribute Infrastructure:}
  \begin{itemize}
    \item \textbf{Public Cloud Credits:} Financial contribution for shared infrastructure
    \item \textbf{Vendor-Managed Runner Pools:} Tightly integrated with GitHub Actions
    \item \textbf{Vendor-Managed Test Infrastructure:} Loosely integrated with GitHub
  \end{itemize}
  \vspace{0.5em}
  \centering
  \scriptsize
  \renewcommand{\arraystretch}{1.3}
  \setlength{\arrayrulewidth}{0.4pt}
  \begin{tabular}{l@{\hspace{0.5em}}>{\raggedright\arraybackslash}p{0.24\textwidth}|>{\raggedright\arraybackslash}p{0.24\textwidth}|>{\raggedright\arraybackslash}p{0.24\textwidth}}
    & \multicolumn{1}{>{\raggedright\arraybackslash}p{0.24\textwidth}}{\scriptsize\textbf{\textcolor{custom1}{Community}}} & \multicolumn{1}{>{\raggedright\arraybackslash}p{0.24\textwidth}}{\scriptsize\textbf{\textcolor{custom1}{Scale}}} & \multicolumn{1}{>{\raggedright\arraybackslash}p{0.24\textwidth}}{\scriptsize\textbf{\textcolor{custom1}{Security}}} \\
    \noalign{\hrule height 0.4pt}
    \multicolumn{1}{|l@{\hspace{0.5em}}}{\scriptsize\textbf{\textcolor{custom1}{Advantages}}} &
    \multicolumn{1}{|>{\raggedright\arraybackslash}p{0.24\textwidth}|}{\cellcolor{custom1!15}\scriptsize Flexible models accommodate different vendor capabilities} &
    \cellcolor{custom1!15}\scriptsize Outsourcing engineering with vendor-managed setups &
    \multicolumn{1}{>{\raggedright\arraybackslash}p{0.24\textwidth}|}{\cellcolor{custom1!15}\scriptsize More loosely coupled systems cannot compromise core CI system} \\
    \noalign{\hrule height 0.4pt}
    \multicolumn{1}{|l@{\hspace{0.5em}}}{\scriptsize\textbf{\textcolor{custom1}{Challenges}}} &
    \multicolumn{1}{|>{\raggedright\arraybackslash}p{0.24\textwidth}|}{\cellcolor{custom1!15}\scriptsize Hide the underlying infra complexity to contributors} &
    \cellcolor{custom1!15}\scriptsize Consistent tech stack required for public cloud credits. Consistent metrics and monitoring across the board. &
    \multicolumn{1}{>{\raggedright\arraybackslash}p{0.24\textwidth}|}{\cellcolor{custom1!15}\scriptsize Ensure security best practices on vendor managed infrastructure} \\
    \noalign{\hrule height 0.4pt}
  \end{tabular}
  \renewcommand{\arraystretch}{1}
\end{grayframe}
\note[itemize]{
  \item Three main contribution types
  \item Credits: for vendors who prefer financial contribution
  \item Runner Pools: vendors provision and maintain runners with GitHub integration
  \item Test Infrastructure: vendors provide test environments with looser coupling
  \item Flexibility helps vendors participate
  \item Each model addresses the scale, security, and experience challenges
  \item Technical details will be covered in architecture section
}

\begin{lblackrwhiteframe}
  \frametitle{Roles \& Requirements}
  \vspace{3em}
  \begin{columns}[c]
    \column{0.55\textwidth}
    \textbf{Least privilege access to cloud resources:}
    \begin{itemize}
      \item Jobs specify nodes access level required
      \item Runners are assigned a role
      \item Roles define the level of access to resources
    \end{itemize}
    \vspace{0.5em}
    \textbf{Requirements for Runners:}
    \begin{itemize}
      \item Configuration
      \item Provisioning
      \item Monitoring
      \item Security
    \end{itemize}

    \column{0.45\textwidth}
    \centering
    \raisebox{-0.5\height}{\includegraphics[width=1.1\textwidth,height=0.6\paperheight,keepaspectratio]{img/pytorch-multi-cloud-infra-Roles.png}}
  \end{columns}
\end{lblackrwhiteframe}
\note[itemize]{
  \item Different roles have different requirements
  \item Testers can tolerate some downtime, releasers cannot
  \item Security requirements scale with privilege level
  \item Publishers and releasers need highest security (access to secrets, signing keys)
  \item All roles require monitoring and audit logging
  \item SLOs help vendors understand expectations
  \item Clear requirements enable better planning
}

\begin{lwhiterblackframe}
  \frametitle{Triggers \& SLOs}
  \begin{columns}[c]
    \column{0.45\textwidth}
    \centering
    \includegraphics[width=1.1\textwidth,height=0.6\paperheight,keepaspectratio]{img/diagrams-SLOs.png}
    
    \column{0.55\textwidth}
    \small
    Trigger levels for CI/CD jobs:
    \begin{itemize}
      \item On-demand only
      \item Periodic nightly
      \item Periodic multi-times per day
      \item Specific PRs only
      \item All PRs
    \end{itemize}
    ~\\
    Service Level Objectives:
    \begin{itemize}
      \item Availability \%
      \item Reliability (Num of infra failures)
      \item Average Job Queue Time (p95/3 months)
      \item Engineering Commitment (on-call, slack, meetings)
    \end{itemize}
  \end{columns}
\end{lwhiterblackframe}
\note[itemize]{
  \item Different roles have different requirements
  \item Testers can tolerate some downtime, releasers cannot
  \item Security requirements scale with privilege level
  \item Publishers and releasers need highest security (access to secrets, signing keys)
  \item All roles require monitoring and audit logging
  \item SLOs help vendors understand expectations
  \item Clear requirements enable better planning
}

\begin{whitetoptealframe}
  \frametitle{Contribution Process}
  \vspace{0.5em}
  \centering
  \includegraphics[width=0.65\textwidth,keepaspectratio]{img/diagrams-Process.png}
  
  \vspace{-0.5em}
  \begin{columns}[t]
    \column{0.48\textwidth}
    \small
    \begin{itemize}
      \item Working Group provides guidance and recommendations
      \item Infra team verify checks for compliance
      \item TAC Community for final approval
    \end{itemize}
    
    \column{0.48\textwidth}
    \small
    \begin{itemize}
      \item Transparent process with clear requirements
      \item Monitoring for SLOs
    \end{itemize}
  \end{columns}
\end{whitetoptealframe}
\note[itemize]{
  \item Clear process reduces friction
  \item Not just "plug and play" - vendors must meet standards
  \item Security review is mandatory
  \item Start small, prove reliability
  \item Gradual rollout reduces risk
  \item Maintainers represent project interests
  \item Vendors can voice concerns but can't veto
  \item Transparency in decision rationale
  \item All decisions documented and public
  \item Conflicts will happen - plan for them
  \item Maintainers represent project interests
  \item Vendors can voice concerns but can't veto
  \item Transparency in decision rationale
  \item Vendors can leave if unhappy (and vice versa)
}

\begin{grayframe}
  \frametitle{Central Administration \& Visibility}
  \begin{columns}[c]
    \column{0.48\textwidth}
    {\footnotesize
    Maintaining Manageability at Scale:
    \begin{itemize}
      \item Common software stack across clouds\\Reusable by vendors too
      \item Self-service Tools with Central Admin Override
      \item Vendor playbooks, reusable IaaC
    \end{itemize}
    ~\\
    Visibility \& Control:
    \begin{itemize}
      \item Public Dashboards: Transparency for community and vendors
      \item Cost Tracking: Attribution per vendor and workload type
      \item Performance Metrics: Build times, success rates, SLA compliance
    \end{itemize}
    }
    
    \column{0.50\textwidth}
    \centering
    \includegraphics[width=0.98\textwidth,height=0.25\paperheight,keepaspectratio]{img/hud.png}
    \vspace{0.4em}
    
    \includegraphics[width=0.98\textwidth,height=0.25\paperheight,keepaspectratio]{img/gharts1.png}
    \vspace{0.4em}
    
    \includegraphics[width=0.98\textwidth,height=0.25\paperheight,keepaspectratio]{img/gharts3.png}
  \end{columns}
\end{grayframe}
\note[itemize]{
  \item System must remain manageable as it grows
  \item Standardization is key to scalability
  \item Automation reduces maintenance burden
  \item Centralized monitoring provides visibility
  \item Public dashboards build trust with community
  \item Track performance: build times, success rates
  \item Understand true costs of each platform
  \item Hold vendors accountable to SLAs
  \item Comprehensive logging for accountability
  \item Self-service reduces bottlenecks
}

% SECTION 4: TECHNICAL ARCHITECTURE
\section[Technical Architecture]{Technical Architecture}

\begin{sectionwithpiclargecentral}[nadi-whatisdelirium-fZ8uf_L52wg-unsplash.jpg]{Photo by \href{https://unsplash.com/@whatisdelirium}{\underline{Nadi Whatisdelirium}}, CC0}{1}
\end{sectionwithpiclargecentral}

\begin{tblackbgrayframe}{High Level Architecture}
  \vspace{1.5em}
  \begin{center}
    \includegraphics[width=0.95\textwidth,height=0.80\textheight,keepaspectratio]{img/pytorch-multi-cloud-infra-Overview.png}
  \end{center}
\end{tblackbgrayframe}

\begin{lgrayframerpicscaled}[pytorch-multi-cloud-infra-Infrastructure.png]{}{%
  \frametitle{Public Cloud Infrastructure}
  \small
  \textbf{Current AWS Setup:}
  \begin{itemize}
    \item Primary infrastructure hosted on AWS
    \item Two accounts, funded by AWS Credits and Meta
    \item Self-hosted GitHub Actions runners
    \item Managed by PyTorch Infra team
    \item Autoscaling based on workload demand
  \end{itemize}
  ~\\
  \textbf{Multi-Cloud \& Infra Working Group Initiatives:}
  \begin{itemize}
    \item \textbf{Portable CI Jobs:} Run in a container, remove cloud-specific assumptions
    \item \textbf{Reusable Autoscaler:} Developing portable autoscale based on ARC
    \item \textbf{Infrastructure as Code:} Reusable OpenTofu modules
  \end{itemize}
}{0.50}{0.45}
\end{lgrayframerpicscaled}
\note[itemize]{
  % TODO here on in-slide, link back to original challenges
  \item AWS is the foundation, but not the only option
  \item Multi-cloud WG working to make system portable
  \item Autoscaler being generalized for other clouds
  \item CI jobs being standardized to run anywhere
  \item IaC approach enables reproducibility
  \item Close collaboration between WG and infra team
  \item Goal: reduce vendor lock-in while maintaining stability
}

\begin{lwhiterblackframeconfig}{0.50}
  \frametitle{Vendor-Managed Runner Pools}
  \begin{columns}[T]
    \column{0.46\textwidth}
    \small
    \settextcolour{myblack}
    \textbf{Challenges:}
    \begin{itemize}
      \item Long lived admin credentials
      \item Runner metadata cannot be enforced
      \item Lack of central administration
      \item Lack of auditability
    \end{itemize}
    
    \column{0.42\textwidth}
    \small
    \settextcolour{white}
    \textbf{Proposed Solution:}
    \begin{itemize}
      \item GHA Runner Token Service (GHARTS)
      \item Use LFID for authentication
      \item No Admin Credentials to vendors
      \item Ephemeral, non reusable credentials
      \item Enforcement of metadata and quota
      \item Centralized administration
      \item Auditability
    \end{itemize}
  \end{columns}
\end{lwhiterblackframeconfig}
\note[itemize]{
  \item GHARTS solves key security and scalability challenges
  \item Traditional approach uses long-lived tokens (security risk)
  \item GHARTS provides short-lived tokens with proper scoping
  \item Centralized control while vendors manage infrastructure
  \item Automated lifecycle reduces manual overhead
  \item All runner activity logged for audit
  \item Standard interface makes vendor onboarding easier
  \item Enables trust but verify model
}

\begin{whitetoptealframeconfig}{0.30}
  \frametitle{GHARTS Provisioning Flow}
  \vspace{1.5em}
  \centering
  \includegraphics[width=0.65\textwidth,keepaspectratio]{img/jit_flow-JIT Runner Provisioning Flow Demo.png}
  
  \vspace{2em}
  \begin{columns}[t]
    \column{0.48\textwidth}
    \scriptsize
    \begin{itemize}
      \item Vendor authenticates using existing credentials (LFID)
      \item Vendor requests a JIT config using the JWT
      \item GHARTS verifies AuthN and AuthZ for vendor
    \end{itemize}
    
    \column{0.48\textwidth}
    \scriptsize
    \begin{itemize}
      \item GHARTS requests JIT config from GitHub
      \item Vendor uses JIT config to provision a runner
      \item JIT lasts 1h, and can only be used Once
    \end{itemize}
  \end{columns}
\end{whitetoptealframeconfig}
\note[itemize]{
  \item GHARTS is the central control plane
  \item Token service is the security foundation
  \item Short-lived tokens (minutes to hours, not days/months)
  \item Policy engine enforces who can run what workloads
  \item Metrics collector provides visibility
  \item Automated lifecycle reduces operational burden
  \item Vendors can't bypass security controls
  \item System scales to many vendors and runners
}

\begin{lblackrwhiteframe}
  \frametitle{Vendor-Managed Test Systems}
  \vspace{3em}
  \begin{columns}[c]
    \column{0.55\textwidth}
    \small
    \textbf{Accelerator Integration Working Group Initiative:}
    \begin{itemize}
      \item Loosely coupled with PyTorch CI
      \item Events relayed to external repository
      \item Vendors provide test infrastructure and environments
      \item Out-of-tree test workflows
    \end{itemize}
    \vspace{0.5em}
    \textbf{Use Cases:}
    \begin{itemize}
      \item Specialized hardware testing (TPUs, custom accelerators)
      \item Produce test results only
    \end{itemize}

    \column{0.45\textwidth}
    \centering
    \raisebox{-0.5\height}{\includegraphics[width=1.0\textwidth,height=0.6\paperheight,keepaspectratio]{img/accelerators-integration.png}}
  \end{columns}
\end{lblackrwhiteframe}
\note[itemize]{
  \item Different model from GitHub runners
  \item Looser coupling gives vendors more flexibility
  \item Good for specialized hardware not suitable for runners
  \item Vendors control their test environment completely
  \item CI orchestrates but doesn't execute directly
  \item Results still integrated into PR workflow
  \item Lower barrier to entry for some vendors
  \item Accelerator Integration WG created this process
}

\begin{grayframe}
  \frametitle{Test System Visibility}
  \textbf{Challenges:}
  \begin{itemize}
    \item Loosely coupled systems harder to monitor
    \item Results come from external systems
    \item Need consistent reporting format
    \item Debugging failures more complex
  \end{itemize}
  ~\\
  \textbf{Ongoing Work:}
  \begin{itemize}
    \item Standardized result reporting via GitHub workflow\_run events
    \item Exploring alternatives: OpenTelemetry for metrics, CDEvents for event streaming
    \item Unified dashboard for cross-vendor visibility
    \item Real-time event relay for debugging
  \end{itemize}
\end{grayframe}
\note[itemize]{
  \item Visibility is the main challenge with loose coupling
  \item Hard to see what's happening in vendor systems
  \item Working on standardized reporting
  \item Event relay provides real-time updates
  \item Dashboard gives unified view across all vendors
  \item Metrics help track reliability and performance
  \item Debugging tools help when things go wrong
  \item Goal: maintain flexibility while improving visibility
  \item Balance between vendor autonomy and project needs
}

% SECTION 5: RESULTS & LESSONS
\section[Results \& Next Steps]{Results \& Lessons Learned}

\begin{sectionwithpiclargecentral}[julie_falk_flickr_22258190324_6a583208ae_k.png]{Photo by \href{https://www.flickr.com/photos/piper/}{\underline{Julie Falk}}, CC BY-NC 2.0}{1}
\end{sectionwithpiclargecentral}

\begin{grayframe}
  \frametitle{Platform Coverage Expansion}
  \textbf{Achievements:}
  \begin{itemize}
    \item \textbf{Public Cloud Credits:} AWS, Meta funding
    \item \textbf{Runner Pools:} IBM (via GHARTS), additional vendors onboarding
    \item \textbf{Vendor Systems:} AMD, Intel, NVIDIA, ARM for specialized accelerator testing
    \item Significant increase in platform coverage across CPU architectures and accelerators
    \item Reduced time-to-market for new platform support
    \item Improved test reliability through dedicated infrastructure
  \end{itemize}
\end{grayframe}
\note[itemize]{
  \item Quantify the success - use real numbers
  \item Specific examples of new platforms added
  \item Vendors can add support faster than before
  \item Better infrastructure = more reliable tests
}

\begin{grayframe}
  \frametitle{Vendor Onboarding Experience}
  \textbf{Positive Feedback:}
  \begin{itemize}
    \item Clear expectations and requirements documentation
    \item Reasonable onboarding timeline with good support
    \item Responsive technical support from maintainers
    \item Fair and transparent governance model
  \end{itemize}
  ~\\
  \textbf{Ongoing Initiatives:}
  \begin{itemize}
    \item IBM trialing GHARTS for runner pool integration
    \item Red Hat collaborating on vendor system visibility improvements
    \item Vendors actively sharing experiences and best practices
  \end{itemize}
\end{grayframe}
\note[itemize]{
  \item Vendors generally positive
  \item Process is clear and documented
  \item Maintainers are helpful
  \item Governance feels fair
  \item Be honest about challenges faced
}

\begin{grayframe}
  \frametitle{Key Lessons Learned}
  \textbf{Principles for Collaborative Infrastructure:}
  \begin{itemize}
    \item \textbf{Fair Governance:} Protect everyone's interests, vendors can't veto but have voice
    \item \textbf{Engage Vendors Early:} Let them help build solutions, not just consume them
    \item \textbf{Empower Maintainers:} Give infra team authority and tools to manage at scale
    \item \textbf{Security First:} Build security in from day one, not as afterthought
    \item \textbf{Flexible Models:} Multiple contribution paths reduce friction and attrition
  \end{itemize}
  ~\\
  \textbf{Challenges:}
  \begin{itemize}
    \item Balancing vendor needs with project sustainability
    \item Standardization across diverse platforms and requirements
  \end{itemize}
\end{grayframe}
\note[itemize]{
  \item Governance clarity prevents conflicts
  \item Security can't be an afterthought
  \item Flexibility helps vendors participate
  \item Transparency builds trust
  \item Balance is ongoing challenge
  \item Standardization requires compromise
}

\begin{grayframe}
  \frametitle{Applicability to Other LF Projects}
  \textbf{Shared Challenges:}
  \begin{itemize}
    \item Accepting vendor infrastructure
    \item Maintaining project neutrality
    \item Ensuring transparency
    \item Achieving financial sustainability
  \end{itemize}
  ~\\
  \textbf{This Model Can Help:}
  \begin{itemize}
    \item Proven governance framework
    \item Technical architecture patterns
    \item Onboarding playbook
  \end{itemize}
\end{grayframe}
\note[itemize]{
  \item Many LF projects face similar issues
  \item Vendor contributions are common need
  \item Neutrality is critical for LF projects
  \item This model is reusable
  \item Share learnings with community
}

\begin{grayframe}
  \frametitle{Future Directions}
  \textbf{Ongoing Activities:}
  \begin{itemize}
    \item Complete GHARTS rollout with IBM and additional vendors
    \item Enhance monitoring and observability across all integration models
    \item Expand portable CI/CD tooling for multi-cloud deployments
    \item Standardize event streaming with OpenTelemetry/CDEvents
  \end{itemize}
  ~\\
  \textbf{Long-term Goals:}
  \begin{itemize}
    \item Share governance and technical patterns with other LF projects
    \item Build community of practice around collaborative infrastructure
    \item Continuously refine based on vendor and maintainer feedback
  \end{itemize}
\end{grayframe}
\note[itemize]{
  \item Model is working, now scale it
  \item More automation reduces overhead
  \item Help other projects adopt this
  \item Governance will evolve
  \item Create community around this approach
}

\begin{grayframe}
  \frametitle{Key Takeaways}
  \begin{itemize}
    \item \textbf{Sustainable Scaling:} Distribute infrastructure costs while maintaining project control
    \item \textbf{Governance Matters:} Fair rules enable vendor participation without compromising autonomy
    \item \textbf{Security \& Transparency:} Non-negotiable foundations that build trust
    \item \textbf{Flexible Models:} Multiple contribution paths reduce friction and increase adoption
    \item \textbf{Reusable Pattern:} This model can help other Linux Foundation projects
  \end{itemize}
\end{grayframe}
\note[itemize]{
  \item Summarize main points
  \item Emphasize that this works
  \item Governance is key to success
  \item Security and transparency build trust
  \item Encourage others to try this model
}

% SECTION 6: Q&A
\section[Q\&A]{Questions?}

\begin{sectionwithpiclargecentral}[carl-jorgensen-5nrnxx_tWe8-unsplash.jpg]{Brecon Beacons, Wales, Photo by \href{https://unsplash.com/@scamartist}{\underline{Carl Jorgensen}}, CC0}{1}
\end{sectionwithpiclargecentral}

\begin{blackframe}
  \frametitle{References \& Contact}
  \textbf{Resources:}
  \begin{itemize}
    \item PyTorch Foundation: \url{https://pytorch.org/foundation}
    \item PyTorch CI/CD: \url{https://github.com/pytorch/pytorch/wiki/CI}
    \item GitHub: \url{https://github.com/pytorch/pytorch}
    \item Multi-Cloud Infrastructure WG: PyTorch Slack \#infra-wg
  \end{itemize}
  ~\\
  \textbf{Contact:}
  \begin{itemize}
    \item Andrea Frittoli
    \item andrea.frittoli@uk.ibm.com
    \item \faXTwitter ~@blackchip76 | \faGithub ~afrittoli
  \end{itemize}
\end{blackframe}
\note[itemize]{
  \item Provide resources for follow-up
  \item Make yourself available for questions
  \item Encourage people to reach out
}

\end{document}
